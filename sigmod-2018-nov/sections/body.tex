
\section{Related Work} \label{related-work}
Traditional machine learning systems focus solely on training models and leave the task of deploying and maintaining these models to the users.
It has only been recently that some systems, for example Velox \cite{crankshaw2014missing}, TensorFlow Serving \cite{abadi2016tensorflow}, and LongView \cite{akdere2011case} have proposed architectures that also consider model deployment and query answering.
LongView integrates predictive machine learning models into relational databases. 
It answers predictive queries and maintains and manages the models.
LongView uses techniques such as query optimization and materialized view selection to increase the performance of the system.
However, it only works with batch data and does not provide support for real-time queries. 
As a result it does not support incremental learning.
In contrast, our system is designed to work in a dynamic environment where it answers prediction queries in real-time and incrementally updates the model when required.
TensorFlow Serving provides mechanisms for real-time queries, deployment and version control of machine learning models.
It has out-of-the-box support for models created using TensorFlow and provides several interfaces for users to deploy their custom models.
However, it does not provide incremental updates to the model.
Contrary to our system, models have to be retrained outside of the system and have to be redeployed to TensorFlow Serving once the training is complete.
Our system supports incremental and batch updates to the model and automatically applies these updates to the model currently being served.

Velox is an implementation of the common machine learning serving practice \cite{crankshaw2014missing}, explained in Section \ref{introduction}.
Velox supports incremental learning and can answer prediction queries in real-time.
It also eliminates the need for users to manually retrain the model offline and redeploy it again.
Velox monitors the error rate of the model using a validation set.
Once the error rate exceeds a predefined threshold, Velox initiates a complete retraining of the model using Spark. 
This deployment method, however, has three drawbacks; retraining discards updates that have been applied to the model so far, the process of retraining on full data set is resource intensive and time consuming, and new datasets introduced to the system only influence the model after the next retraining.
Our approach differs, as it exploits the underlying properties of SGD to fully integrate the training process into the system's lifeline.
This eliminates the need for completely retraining the model and replaces it with consecutive SGD-iterations.
Moreover, our system can train the model on new batch datasets as soon as they become available.

Clipper \cite{crankshaw2016clipper} is another machine learning deployment system that focuses on producing higher quality predictions by maintaining an ensemble of models.
It constantly examines the confidence of each model.
For each prediction request, it uses the model with the highest confidence.
However, it does not incrementally train the models in production, which over time leads to models becoming outdated.
Our deployment method on the other hand, focuses on maintenance and continuous updates of the models.

Weka \cite{hall2009weka}, Apache Mahout \cite{Owen:2011:MA:2132656}, and Madlib \cite{hellerstein2012madlib} are systems that provide the necessary toolkits to train machine learning models. 
All of these systems provide a range training algorithms for machine learning methods. 
However, they do not provide any management, before or after the models have been deployed. 
Our proposed system focuses on models trainable using stochastic gradient descent and as a result is able to provide model management both during training and deployment time.

MLBase \cite{kraska2013mlbase} and TuPaq \cite{sparks2015tupaq} are model management systems.
They provide a range of training algorithms to create machine learning models and mechanism for model search as well as model management.
They focus on training high quality models by performing automatic feature engineering and hyper-parameter search.
However, they only work with batch datasets.
Once models are trained, they have to be deployed and used for serving manually by the users.
Our system, on the contrary, is designed for deployment and maintenance of already trained models.


\section{Continuous Training and Serving} \label{continious-training-serving}
Our proposed deployment and maintenance system uses SGD as its underlying learning algorithm.
As a result, it can update the model incrementally (one training item at a time) or use mini-batches of data.
\todo[inline]{schelter: awful?}
\subsection{Stochastic Gradient Descent} \label{sgd}
\todo[inline]{Schelter: rewrite!!}
Machine learning minimizes an objective function (often referred to as the loss function).
Usually, this requires us to calculate the gradient of the function at different data points and update the function parameters based on the gradient values.
A common optimization method is gradient descent, an iterative process where each iteration uses the entire training data set to calculate the gradient value.
One drawback of gradient descent is that in presence of large datasets it performs very slow.
Stochastic gradient descent \cite{bottou2010large} is an approximation of the gradient descent method. 
Similar to gradient descent, it is an iterative process.
However, in each iteration it calculates the gradient at a single element (or a small sample) and updates the parameters of the model accordingly. 
Although it converges after a higher number of iterations, the overall convergence time is lower (sometimes by orders of magnitude) than gradient descent \cite{bottou2010large}. 
Each iteration of SGD executes in a short amount of time because it only works with a sample of the data.
We are leveraging this property of SGD and design our deployment system so that it performs one iteration of SGD at a time without interrupting the query answering component.
In Section \ref{evaluation}, we show that the overhead from executing a single iteration of SGD is very small and does not affect the query answering component.

\textbf{Distributed SGD. }
To efficiently train machine learning models on large datasets, scalable techniques have to be employed.
SGD inherently works well with large amounts of data because it does not need to scan every data point during every iteration.
However, for very large datasets, SGD has to perform many iterations in order to converge.
To decrease the running time, large datasets can be distributed among multiple nodes, where each node will compute the gradients on a subset of the data in parallel.
One drawback of this approach is that a synchronization step is required before applying the updates to the model, which slows down the optimization process.
To alleviate this, several asynchronous SGD methods are proposed \cite{recht2011hogwild, dean2012large}. 
\hl{Experiments using these methods show that the quality of the produced model in some cases is similar to the synchronized SGD approach.}

\textbf{Machine Learning Models based on SGD. }
SGD is a common optimization methods that has been used in classification \cite{zhang2004solving}, clustering \cite{bottou1995convergence}, neural networks \cite{dean2012large}, and matrix factorization \cite{funk2006netflix}.
Some examples of machine learning models that use SGD are: 

\textbf{Linear Classifiers} are arguably the most common type of machine learning models built using the SGD optimization method. 
In the CTR prediction example of Section \ref{introduction}, we use logistic regression to train the model for predicting the click through rate \cite{macmahan2013}. 
Logistic regression models typically output a probability instead of a class label that indicates how likely an item belongs to a specific class \cite{hosmer2013applied}.
In our example, logistic regression predicts the click probabilities for all the available advertisements and the ones with the highest probabilities are displayed.
Support vector machines (SVM) represent another common class of classification models \cite{steinwart2008support}.

\textbf{Matrix Factorization} is a common method used in recommender systems \cite{koren2009matrix}. 
Matrix factorization is used to derive the latent factors (e.g., for users and items) for recommender systems.
It relies on the fact that each user and item can be described in a few dimensions (10 to 40 usually) based on the available ratings.
These latent factors automatically capture the similarity of users and items based on the ratings provided by the users.
Hence, any unknown rating can be predicted by computing the dot product of the user vector with the item vector.

\textbf{Neural Networks} or deep learning -- inspired by biological neural networks in the brain -- are used to learn and approximate complex functions. 
They have been used for more than half a century to model functions and have been successfully applied to training machine learning models.
However, due to the slow training process and the lack of large amount of training data, they have not been used extensively in the machine learning community in the past.
In the last decade, there was a drastic change due to several seminal publications.
Hinton et al. proposed methods for speeding up the training of neural networks \cite{hinton2006fast}.
The ImageNet competition \cite{ILSVRC15} in 2012 was won by a neural network proposed by Krizhevsky et al. where they significantly reduced the error rate \cite{krizhevsky2012imagenet}. 
The success of the Google's Deepmind team in achieving Neural Networks that were capable of defeating humans in the game of Go \cite{silver2016mastering} and mastering Atari games \cite{mnih2013playing} was also instrumental in popularizing neural networks in the machine learning community.

SGD is the ideal optimization algorithm for training neural networks since it works very well with large datasets (which are required for training neural networks).
In fact, almost all recent work on neural networks uses SGD for training them.
\todo[inline]{Schelter: too much blabla, not ....}

\begin{figure}[t]
\centering
\includegraphics[width=\columnwidth]{../images/improved-example.pdf}
\caption{Continuous Training and Serving}
\label{fig:cont-training-serving}
\end{figure}
The core design principles of our deployment system are threefold.
First, we incrementally update the model so that it can adapt to changes in the distribution of the incoming data.
Second, we eliminate retraining and replace it with a series of consecutive iterations of SGD.
And finally, we immediately use new batch datasets that are available to the system.
Figure \ref{fig:cont-training-serving} shows how our deployment method works.
First, using the existing data residing on disk, we train an initial model and deploy it into the system.
The system receives prediction queries and training observations in a streaming fashion.
The deployed model answers incoming prediction queries as soon as they are received.
Once the system receives a training observation it updates the model incrementally.
The system also keeps track of incoming training observations and adds them to an intermediate buffer.
A scheduler component, triggers new iterations of SGD based on the rate of incoming training observations. 
The scheduler can also decide to run an iteration when the system is not under heavy load.
Each new iteration uses a random sample of the data in storage and the data in the buffer. 
Moreover, our system stores new batch datasets in the buffer (or the persistent storage unit) as soon as they become available.
Any further scheduled iteration of SGD incorporates the new data without requiring the model to be retrained from scratch.

\todo[inline]{R3: Section 5 has many unclear and informal descriptions, without formal methodology.}
\section{System Architecture} \label{sec:system-architecutre}
\todo[inline]{explain the statistics collection unit and the scheduler (in more detail)}
\begin{figure}[t]
\centering
\includegraphics[width=\columnwidth]{../images/system-architecture-final.pdf}
\caption{System Architecture}
\label{fig:system-architecture}
\end{figure}

The proposed system comprises three main components; model manager, data manager and scheduler, and an independent SGD run-time. 
Figure \ref{fig:system-architecture} gives an overview of the architecture of our system and the interactions among its components.
The data manager first stores the incoming training observations in a buffer and then passes them on to the model manager.
The model manager incrementally updates the model using the training observations.
The model manager is also responsible for receiving prediction requests.
Once it receives a request, it uses the latest version of the model to make a prediction and returns the result to the user.
Both the scheduler and the data manager components are constantly communicating with each other and with the model manager, to obtain the latest statistics, such as the model quality and buffer size.
This in turn helps us to tune the scheduling and sampling rate for future iterations of SGD. 
Next, we explain each component of the system in more detail.

\subsection{Scheduler}\label{scheduler}
The scheduler component is responsible for scheduling new iterations of SGD.
Intuitively, the best time to execute an iteration is when the system is not under heavy load.
A new iteration of SGD is also executed when the system receives more training data than can be handled by the intermediate buffer.
If the model is not updated with the new training items frequently, the quality decreases.
This decrease in the quality is more rapid if the distribution of the data is changing.
In our prototype, the scheduling rate is controlled by a user defined parameter, \textit{max\_buffer\_size}.
When the intermediate buffer's size reaches \textit{max\_buffer\_size}, the scheduler executes a new iteration of SGD.
It is important to note that the scheduling rate affects the quality of the model.
In Section \ref{evaluation}, we investigate the effect of scheduling rate on model quality.
If no new training data is available, the model parameters will eventually converge and any further training iterations will not have any effect on the model quality.
\todo[inline]{R3: How is the convergence detected?}
Therefore, the scheduler component has to communicate with the model manager in order to detect whether the model parameters have converged and stop further iterations until more training data becomes available.

\subsection{Data Manager} \label{data-manager}
In order to execute an iteration of SGD, we need to combine the training data that arrives at the system in real-time with the data stored on disk.
The data manager is responsible for storing the incoming training observations in an intermediate buffer.
Upon a new training iteration, the data manager accesses the historical data stored on disk and provides a sample.

Different sampling strategies can be used to provide the sample.
In our current prototype, the data manager uses a simple unified random sampling method to generate this sample.
More advanced methods, such as Reservoir \cite{vitter1985random} or weighted random sampling can also be used to generate the sample.
Reservoir sampling is typically used to generate samples from large datasets that do not fit in memory, whereas weighted random sampling is used when data elements have different weights.
In an online machine learning scenario, recent items are more important for training the model and are assigned a bigger weight than older items.
Therefore, weighted random sampling can generate samples that can contribute to the training of a better model.

The data from the sample and the data in the buffer are merged to create the dataset for next training iteration.
The data manager provides access to this dataset for the model manager in order to further train the model.
The data manager also communicates with the scheduler in order to inform it when the intermediate buffer is becoming full and a new training iteration is required. 

The created data set consists of the data inside the buffer and a sample of the historical data as described earlier.
The sampling rate is a system parameter that has to be configured.
It can be pre-configured to a constant value based on the application.
However, using the feedback from the system's model manager (Section \ref{model-manager}), the sampling rate can be adjusted.
For example, when the data distribution is changing, a smaller sampling rate places more emphasis on the data that arrived recently. 
This is similar to the problem of concept drift where the distribution of the incoming data changes overtime.
This renders historical data less important and as a result a smaller sample of the historical data (or none at all) will give more importance to the data in the buffer and help the model to adopt faster to the concept drift.
However, if there is no concept drift in the data, a larger sampling rate will increase the quality of the model after a training iteration.
Another effect that the sampling rate has on the system is the training iteration running time.
A larger sampling rate increases the running time of each training iteration as more data has to be processed.
In Section \ref{evaluation}, we investigate the effects of different sampling rates on both the quality and performance of the system.

Moreover,  new data sets can be registered in data manager.
In our current prototype, new data sets first have to be stored on disk, and data manager can be informed of the data path.
Newly available data sets are used in the subsequent SGD iterations.

\todo[inline]{R2: The difference of incremental update and SGD update needs clarification. In the experiment, they are used in different approaches (baseline+ is using incremental update and Continuous is using SGD). But, the second sentence of the last paragraph of sec 5.3 says both updates are applied in Continuous. BD: I have used incremental to refer to online learning. Whereas in some literature incremental may also refer to mini batch updates. I think the reviewers comment is already addressed in section \ref{subsec:setup}}
\subsection{Model Manager} \label{model-manager} 
An important part of the system is the model manager component.
It is responsible for storing the model, answering prediction queries, and performing incremental and batch updates to the model.
Listing \ref{model-manager-api} shows the API of the model manager.
The API is used to interact with other components as well as end-users of the system.
The scheduler component uses \textit{update} and \textit{update\_iteration} to instruct the model manager to perform incremental or batch updates (one iteration of SGD) to the model.
Upon a new prediction query, the \textit{predict} method is called to provide the end-user with the label of the given input.

\noindent\begin{minipage}[t]{\linewidth}
\begin{lstlisting}[language=java, basicstyle=\small\ttfamily, frame=tb ,columns=fullflexible,
showstringspaces=false,label=model-manager-api,caption=Model Manager API, numberstyle=\tiny]
def update(x,y)

def update_iteration(X,Y)

def predict(x): Label

def error_rate(X_test, Y_test): Double

\end{lstlisting}
\end{minipage}


The \textit{error\_rate} method returns the error rate of the model using the provided test dataset.
As described earlier, constant monitoring of the quality is required in order to adjust the scheduling and sampling rate.
When the error rate is stagnating, the model has converged using the existing data.
Therefore the model manager informs scheduler not to schedule any new iterations until new training observations have arrived at the system.
\todo[inline]{R3: How is this done exactly?}
Similarly as explained in Section \ref{data-manager}, an increase in the error rate may indicate a change in distribution of the data.
As a result, reducing the sampling rate will place more emphasis on recent data (in the intermediate buffer) and help adapt the model to the changes in the distribution.

The model manager also keeps track of the changes that are made to the model.
The model is updated both through incremental learning and SGD-iteration.
The model manager creates snapshots of the model in two different scenarios; after a series of incremental updates are made and after each training iteration.
\todo[inline]{R3: What is the algorithm here?}
This versioning of the models is essential.
When there is a rapid change in the distribution of the incoming data (a sudden concept drift) or when there are anomalies in the data, it is sometimes necessary to revert back to a version before the change in distribution occurred.
In case of concept drift, new training iterations should be scheduled that only use the data in the buffer.
Moreover, in case of anomalies in the data, they have to be identified and discarded before any further model updates could happen.

\subsection{SGD Run-Time} 
All components of our model serving system described so far are not limited to any specific run-time.
We have decoupled the components from the actual run time of the system.
Any system capable of performing incremental and batch SGD updates efficiently are suitable options for our system.
Apache Flink \cite{carbone2015apache} and Apache Spark \cite{zaharia2010spark} are distributed data processing platforms that work with data in memory and have support for iterative algorithms, which makes both of them ideal options for our SGD run-time.
In our current prototype, we are using Apache Spark \cite{zaharia2010spark} as our SGD run-time.

The model manager is the component responsible for communicating with the SGD run-time.
In the current version of our prototype, the model manager requests Spark to perform both incremental and batch updates to the model.
Both types of updates are supported by the built in machine learning library of Spark.
The choice of run-time for SGD slightly influences the data manager as well.
In our prototype, historical data is stored on the Hadoop Distributed File System (HDFS) \cite{shvachko2010hadoop}.

\todo[inline]{R3: The paper has a point that not only the resource consumption is reduced by incremental update, but also the prediction accuracy can be improved due to adapting the model to newly seen data. The system uses a sampling component to use a subset of historical data in each SGD update. However, a non-incremental retraining can also achieve the same adaptation by weighing towards the newly seen data (e.g., learning with concept drifting). Therefore, the ideas of incremental update and adaptation to concept drift should not be confused. The evaluation in this paper does not clearly decouple them. BD: Interesting insight. We have to address this.}